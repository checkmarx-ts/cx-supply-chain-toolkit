\documentclass[a4paper, 11pt, oneside]{book}
\usepackage[a4paper, total={6.5in, 10in}]{geometry}
\usepackage{hyperref}

\usepackage[svgnames]{xcolor}
\usepackage{graphicx}
\usepackage[utf8]{inputenc}
\usepackage[T1]{fontenc}
\usepackage{PTSerif}
\usepackage{listings}
\usepackage{booktabs}
\usepackage{fouriernc}
\usepackage{makecell}
\usepackage{fontawesome}
\usepackage{hyperref}
\usepackage{tabularx}
\usepackage{xspace}
\usepackage{xparse}
\usepackage{amssymb, amsmath}
\usepackage[many]{tcolorbox}
\tcbuselibrary{listings}


\begin{document}

\begin{titlepage}
    \thispagestyle{empty}
    \centering
    \includegraphics[scale=.4]{graphics/cx_logo-dark.png}
    \vfill
    \textcolor{Sienna}{\Huge Checkmarx Supply Chain Toolkit\\0.0 (Development)}
    \vfill
    {\Large\textbf{Nathan Leach, CSSLP\\Checkmarx Principal Solution Architect}}
\end{titlepage}

\newpage



\chapter*{Quickstart}

If you are not familiar with supply chain scanning and
dependency resolution or want to understand how it works, please
start with the Part \ref{part:background}.


\noindent\\If you are trying to integrate supply chain scanning into
you SDLC, Part \ref{part:operation} covers several related topics:

\begin{itemize}
    \item If you are trying to integrate scanning in an existing CI/CD pipeline by \\
    extension of a container that executes a build stage in your pipeline, \\
    please start with Chapter \ref{chap:ext_build_env}.
    \item If you are trying to automatically select the correct build environment for a \\
    supply chain scan invoked via a web hook, please start with Chapter \ref{chap:build_env_affinity}.
    
\end{itemize}



\tableofcontents

\newtcblisting{xml}[3]{
    listing only,
    title=<#1> #2 #3,
    width=\textwidth,
    listing options={
        basicstyle=\small\ttfamily,
        breaklines=true,
        columns=fullflexible,
    },
}

\newtcblisting{code}[3]{
    listing only,
    title=#1 #2 #3,
    width=\textwidth,
    listing options={
        basicstyle=\small\ttfamily,
        breaklines=true,
        columns=fullflexible,
    },
}


\part{Background}\label{part:background}
\chapter{Scan Integration Challenges}
\section{Understanding Dependency Trees}


Software defines dependencies typically by specifying packages directly
consumed by the software in the build instructions.  These are referred to as
\textit{direct dependencies}.  Each direct dependency may also have
dependencies; dependees of dependencies are referred to as 
\textit{transitive dependencies}.  Each dependency of software can be an
internally developed private package or a publicly available open source package.
\footnote{Commercially developed, closed-source third-party packages are grouped into the
category of open source packages for ease of explanation.}
Figure \ref{fig:dependency_tree} shows an example of a dependency tree.

The dependency tree can have a varying depth depending on the composition
of all dependencies.  If one to review the dependency tree of Hadoop, for
example, a very deep dependency tree with a large variety of packages and
licenses would be observed.  The red boxes in Figure \ref{fig:dependency_tree}
show packages that were detected as having known vulnerabilities.

\begin{figure}[h]
    \caption{Example Dependency Tree}
    \includegraphics[width=\textwidth]{graphics/dependency_tree.png}
    \label{fig:dependency_tree}
\end{figure}


\subsection{How Detection of Vulnerable Dependencies Can Fail}\label{sec:missing_vulnerable_dependencies}


\subsubsection{Dependency Resolution Happy Path}

One of the potentially vulnerable packages in Figure \ref{fig:dependency_tree} 
is an open source direct dependency of the software.  The detection in this
case is very simple since the dependency is specifically referenced in
the software's build script.  The package is open source, making it well known
and generally available for dependency resolution. Provided the tooling that
can resolve the dependency and a network path to the public repository
is available, the dependency can be resolved.


\subsubsection{Unavailable Private Packages}

The other potentially vulnerable package exhibited in Figure 
\ref{fig:dependency_tree} is a transitive dependency of an internally 
developed direct dependency. A prerequisite of discovering a potentially 
vulnerable package that is a dependee of an internally developed package
is that the internally developed package is available at the time of 
dependency resolution.  

It is often the case that an 
internally developed package is not publicly
available; private packages are typically available through private
package repositories hosted on the organizations internal network.
If that private package repository is not accessible over the network,
the private package and its transitive dependencies will not be available.
The challenges presented in detecting vulnerabilities 
in transitive dependencies of private packages are discussed in detail
in Section \ref{sec:execution_environment}.

\subsubsection{Proper Tooling is Unavailable or Misconfigured}

If the tools required to perform a dependency resolution are not available,
it is apparent that the dependency resolution will fail.  Along with
having the correct tools available, the tools need to be correctly configured.
For example, if a private package repository requires login credentials, 
the dependency resolution will fail unless the tooling configuration provides
those login credentials.

\subsubsection{Improperly Defined Direct Dependencies}

Figure \ref{fig:dependency_tree} depicts a dashed line from the software to a
transitive dependency.  When developers are writing software, it is often
possible to reference packages via namespace.  If a referenced package happens to be
pulled in as a node on the dependency tree below a direct dependency, the
developer may see that a compile and execution works due to the dependency
being resolved as part of the build order.  Since the dependency
is referenced directly in the software, it should be defined as a direct
dependency.  Software can exist and build for years without anyone ever realizing
that the build definition is technically wrong.

This often manifests as a confusing issue where supply chain scans produce one
set of results if performed pre-build when compared to the scan performed post-build.
This is because the build order will cause dependencies to properly resolve by
order of reference.  This downloads and locally caches all required dependencies
and makes them available for the supply chain scan's dependency resolution after the
build.

When performed pre-build, the dependency resolution is usually performed in the order
the build steps are defined rather than in the order of build step dependencies.  This means
that it is possible for some of the dependency resolution to fail
since missing dependencies are not in the local cache.

\subsubsection{Cached Deprecated Dependencies}

When using a non-ephermal build environment that would be commonly known as a "build box," 
modification of these environments are often avoided.  It is not uncommon for build box
to have a software environment that is extremely out of date; no one touches it 
since there is not a reason to fix something that isn't broken.  Eventually no one
knows how the build box is configured to work, no one remembers how it needs to be configured,
and all attempts at creating a new build box cause the build to fail.  Most developers
dread build environment changes.

A side effect of a non-ephermal build environment is that each build execution can mutate
the environment.  Package management tools will typically cache packages downloaded as part
of the dependency resolution.  Unless the build tooling specifically prevents this or the
cache is purposefully cleared periodically, the packages typically stay on the build box
for the entire life of the build box.

Package repositories, however, have no reasonable expectation of keeping all historic packages
available in perpetuity.  It is often the case that both private and public package
repositories will purge deprecated packages.  It is also possible the package maintainers
may decide they no longer want to make the packages available and delete the packages
from the package repository.  Since the build works and no one touches the build box, 
any packages no longer available in a package repository will continue to be available in the
build box package cache.  It may take many years before anyone realizes the package
is no longer publicly available.

When the supply chain scan is executed outside of the environment where the package
cache contains the no-longer-available package, there is no way to resolve the package.
This leads to a failure to generate an accurate dependency tree.

Many organizations have moved to ephermal build environments, such as using containers, 
to avoid these issues.  The ephermal environment has a definition that will produce the
environment to be exactly correct each time it is regenerated.  When an ephermal environment
finishes a build, all mutations of the environment are discarded when the ephermal
environment exits.  If using an ephermal build environment, the build will fail immediately
upon trying to retrieve a package that is no longer available.



\section{Execution Environment}\label{sec:execution_environment}

Performing a software build requires the build activities to execute in the
proper execution environment.  The execution environment is often configured
as a "build box" or a container defined as the execution environment in
a pipeline build stage.  The build environment will typically have all the
tools and configuration required to perform a software build intended 
to yield an package 
containing executable software.  As part of the build, the resolution of 
dependencies is typically part of one or more build steps.  This is often 
required for the software to compile and/or to produce
a distributable build output.\footnote{Not all languages require a compile
or to have dependencies available for the distributable packaging.
The hypothetical scenario discussed is that the dependencies need to be
available at some point during software development if they are 
used by the software at runtime.}

Obtaining an accurate dependency resolution requires the dependency tree to be
resolved in the correct execution environment.  This applies to not only the
availability of the tools and configuration required to resolve 
the dependencies, but also to
the network accessibility where the dependency resolution is executed.  It
frequently the case that dependencies are obtained by network services
accessible only when originating a request from an organization's network. 
In cases such as the one illustrated in Figure \ref{fig:dependency_tree},
the vulnerable transitive dependency of an internally developed package may
not be detected in some scenarios.



\subsubsection{Execution Environment Happy Path}\label{sssec:happy_path}

Figure \ref{fig:dependency_resolution_happy_path} illustrates the concept
of a basic dependency resolution happy path.  In the diagram, the supply chain
scan is added to an existing pipeline to perform dependency resolution
and obtain the dependency tree required for the scan.  The dependency tree is
then sent to the scanning service where the tree is assessed for 
vulnerable packages.

The reason this works is that the dependency resolution is executed inside of
the very same environment used to build the software.  Not only is the correct 
tooling available and properly configured, it is run with the same network
accessibility that allows any private packages to be resolved during the build.

A build environment may serve one or more projects that have the same tooling,
configuration, and network accessibility requirements.  As part of configuring
the ability to build the software, the correct environment for the project
is often defined as part of the pipeline definition.  

As an example, refer to the 
\hyperref[listing:ado_pipeline]{\texttt{Azure Devops Pipeline Example}} listing.
It can be observed in the example pipeline definition that the \texttt{container} 
or \texttt{pool} directives can be used to give the pipeline stage execution an affinity 
for a specific build box or container. When the dependency resolution and supply chain scan
is executed, the execution is in an environment that contains all the tools
needed for a successful scan.



\begin{figure}[h]
    \caption{Dependency Resolution Execution Happy Path}
    \includegraphics[width=\textwidth]{graphics/dependency_resolution_happy_path.png}
    \label{fig:dependency_resolution_happy_path}
\end{figure}



\label{listing:ado_pipeline}
\begin{code}{Azure Devops Pipeline Example}{}{}
trigger:
    - master

pool:
    vmImage: ubuntu-latest

jobs:
    - job: SCAResolver
        pool:
            vmImage: ubuntu-latest

        container: 
            image: cxnleach/scaresolver-general-build-ado:latest

        steps:
            - script: /sandbox/resolver/ScaResolver -h
\end{code}




\subsubsection{Generic Execution Environment}\label{sssec:generic_environment}

Figure \ref{fig:dependency_resolution_generic_env} depicts a build pipeline that
submits the code or build definition files to a remote service for dependency
resolution and supply chain scanning.  It is very similar to the scenario
described previously in \hyperref[sssec:happy_path]{\textit{Execution Environment Happy Path}}
with the difference being that the dependency resolution is executing in
an execution environment provided by the remote system.  The remote system may not have the
ability to define a specific execution environment for dependency resolution as can
be done within a pipeline definition.

In this scenario, the generic build environment must have properly installed and configured
all tools required to successfully perform a dependency resolution.  In some cases, 
the tooling installation and configuration requirements is simple and this will work.  There
are other cases where this may not be so simple:

\begin{itemize}
    \item The dependency resolution requires older or newer versions of the build tools.
    \item The build tools are not compatible with the remote platform's build environment.
    \item Internally built or third-party licensed tools are required.
    \item Specific tooling configurations are required to perform the build.
\end{itemize}

If the tooling and configuration is not correctly defined in the generic environment,
the dependency resolution may not accurately produce a dependency tree.

\begin{figure}[h]
    \caption{Dependency Resolution Generic Execution Environment}
    \includegraphics[width=\textwidth]{graphics/dependency_resolution_generic_env.png}
    \label{fig:dependency_resolution_generic_env}
\end{figure}



\subsubsection{Remote Execution Environment}

A remote execution environment is a variation of the 
\hyperref[sssec:generic_environment]{\texttt{Generic Execution Environment}}.  In this
environment, the build environment can be generic or it can have the correct tooling
and configuration necessary to produce a build.  The challenge here is that the network
paths available to the remote execution environment may not allow for an accurate
dependency tree to be generated.

Figure \ref{fig:dependency_resolution} shows the diagram of the network connections
used during the dependency resolution.  The remote environment will generally have
no issues accessing public package repositories to resolve open source direct and transitive
dependencies.  The red line shows an attempted network connection to a package
repository that may be accessible only from within the corporate network.  When the
supply chain scan submits the scan so that dependency resolution occurs in the
remote environment, the network connection may not be available.  

If \texttt{Step A} of the pipeline's \texttt{Stage 2} is executed after the supply chain
scan, it is performed inside the corporate network.  The green line indicates that it can 
reach the internal package repository.  The internal package repository can be reached
since the execution of \texttt{Step A} is performed on the corporate network.

\begin{figure}[h]
    \caption{Dependency Resolution Remote Execution Environment}
    \includegraphics[width=\textwidth]{graphics/dependency_resolution.png}
    \label{fig:dependency_resolution}
\end{figure}







\part{Operation}\label{part:operation}

\chapter{Integrating \scaresolver into Build Environments}\label{chap:ext_build_env}

If you've determined that your dependency resolution scanning does not execute properly
in a \hyperref[sssec:remote_environment]{\textit{Remote Execution Environment}}, \scaresolver
is the solution for executing the dependency resolution in an environment you have defined.
The Checkmarx \scaresolver is the scan command line tool that invokes 
dependency scans locally.  It can be invoked directly or as part of another
tool such as \cxflow, the \cxonecli, or other Checkmarx plugins.

\cxsca is a standalone product that provides a portal that manages supply chain vulnerability scans.  This is
typically used in combination with the \cxsast product to provide both static analysis and supply-chain vulnerability
scans.  \scaresolver can communicate directly with \cxsca to upload data from the locally executed supply-chain
scan, which would have presumably executed in your customized build environment.

\cxone is a product that combines multiple scan types, including supply-chain vulnerability scans, into a single
view.  Scans are typically invoked using the \cxonecli where the type of scan to invoke is defined as part
of the CLI execution parameters.  By default, the \cxonecli will use the 
\hyperref[sssec:remote_environment]{\textit{Remote Execution Environment}} to perform the supply-chain
vulnerability scan.  The \cxonecli can be given a path to the \scaresolver executable to allow
the supply-chain vulnerability scan to execute in your customized build environment.  The \cxonecli the uploads
the results to \cxone for final analysis that reports any potentially vulnerable packages.


\noindent\\This section documents methods for integrating the execution of \scaresolver into your
build environment.


\section{Deploying on a Build Agent}

When builds are executed on a specific build agent (a.k.a. "The Build Box"),
the invocation of \scaresolver will typically be scripted to execute
within a pipeline stage.  In this scenario, the \scaresolver should be
installed on all build agents that will run pipelines that have scripted
a supply chain scan invocation.  This is the most simple scenario and applies
when using \scaresolver with \cxsca or \cxone.

Note that in this scenario, updates to \scaresolver will need to be
periodically installed.  There are likely several tools on the build agent
that need periodic update; \scaresolver will simply be one additional
tool that requires occasional update.

\section{Modifying a Containerized Build Environment}

A variation of the build agent deployment is found when containerized
build environments are defined as the execution environment for
pipeline stages.  The supply chain scan invocation is scripted similar
to how it would be invoked when using a build agent.  The main difference
is that the pipeline stage is configured to execute inside a specified
container image.  The container image contains all the tools required
to successfully build the software, which means that an accurate
dependency tree can also be generated if the \scaresolver is invoked
in that build environment.

In this integration scenario, the container definition can be modified to install \scaresolver
as part of the container build.  The pipeline scripting running in the container
will invoke \scaresolver as it would invoke any other tool in the scripted
build steps.  

\section{Containerized Build Environment Extension}\label{sec:extending_environment}

Often it is difficult to modify the build agent installation or the build
container definition to add \scaresolver.  It may also present some
difficulties in deployment for some pipeline architectures.  Another option
is to create new container instances that derive from already defined build
environments.  This has several advantages:

\begin{itemize}
    \item No instability can be introduced into known-stable build environments.
    \item The \scaresolver updates can be applied without modifying any 
    containerized build environments.
    \item Deployment of updates is a simple rebuild of the extended containers.
    \item The \cxtoolkit provides a way to generate the extended build 
    environment image for build images running on popular Linux distributions.
    \item The image created with the \cxtoolkit also provides some isolation of 
    dependency resolution activities to avoid some attacks associated with 
    malware embedded in typo-squatted packages.
\end{itemize}

\noindent\\The "build-environment" components can be obtained from the \cxtoolkitpath{releases/latest}{Releases}.


\subsection{Creating Extended Images}

The build-environment components contains a \texttt{Dockerfile} is multi-stage where stage names
specify the correct variation of Linux\footnote{There is currently no support for Windows base images.}
in the base image.  The stage names are intended to align 
with popular base images used to create build environments.  The \texttt{Dockerfile} stages
execute the commands specific to the Linux OS distribution of the base image to properly configure
\scaresolver.

Any image that can be pulled from the public Docker Hub or a private docker registry connected via 
\texttt{docker login} can be defined as the base image.  If the wrong or incompatible stage is specified, 
the container build will fail. To see the base image Linux distribution, one possible method would be
to view the \texttt{/etc/os-release} file found in the base image.  This can be done by executing the following
command:

\noindent\\\\\texttt{docker run ----rm -it ----entrypoint=cat <base image tag> /etc/os-release}


\noindent\\As an example, determining the Linux variation for the \texttt{gradle:latest} image can
be performed with the following command:

\noindent\\\\\texttt{docker run --rm -it --entrypoint=cat gradle:latest /etc/os-release}

\noindent\\The output of \texttt{/etc/os-release} reveals that the Linux variation is Ubuntu, which is a derivative
of Debian.\\\\

\begin{code}{Output of "cat /etc/os-release" from gradle:latest}{}{}
PRETTY_NAME="Ubuntu 22.04.3 LTS"
NAME="Ubuntu"
VERSION_ID="22.04"
VERSION="22.04.3 LTS (Jammy Jellyfish)"
VERSION_CODENAME=jammy
ID=ubuntu
ID_LIKE=debian
HOME_URL="https://www.ubuntu.com/"
SUPPORT_URL="https://help.ubuntu.com/"
BUG_REPORT_URL="https://bugs.launchpad.net/ubuntu/"
PRIVACY_POLICY_URL="https://www.ubuntu.com/legal/terms-and-policies/privacy-policy"
UBUNTU_CODENAME=jammy
\end{code}



\subsection{How to Build an Extended Image}

Building the extended image is done via a \texttt{docker build}\footnote{These instructions can likely be adapted to other container build tools.  Docker is used here since it is
the most commonly used container toolkit at the time this manual was written.} command
using the \texttt{Dockerfile} provided in the \cxtoolkit build-environment.
Docker \href{https://docs.docker.com/build/guide/build-args/}{build arguments} control how the extended image build
is performed.  The most important argument is the \hyperref[sec:BASE]{\textbf{BASE}} build argument.  Other build
arguments are available; details of the available arguments can be found in Appendix \ref{chap:build_args}.



\noindent\\Example build commands:

\begin{code}{Extending Gradle 8.0 Alpine with JDK19}{[with Entrypoint]}{}
docker build -t <your tag> --build-arg BASE=gradle:8.0-jdk19-alpine \
    --target=resolver-alpine .
\end{code}

\begin{code}{Extending Gradle 8.0 Alpine with JDK19}{[without Entrypoint]}{}
docker build -t <your tag> --build-arg BASE=gradle:8.0-jdk19-alpine \
    --target=resolver-alpine-bare .
\end{code}

\begin{code}{Extending Node 19 Alpine}{[with Entrypoint]}{}
docker build -t <your tag> --build-arg BASE=node:19-alpine \
    --target=resolver-alpine .
\end{code}

\begin{code}{Extending Node 19 Alpine}{[without Entrypoint]}{}
docker build -t <your tag> --build-arg BASE=node:19-alpine \
    --target=resolver-alpine-bare .
\end{code}
    
\begin{code}{Extending Node 19 Buster (Debian)}{[with Entrypoint]}{}
    docker build -t <your tag> --build-arg BASE=node:19-buster \
        --target=resolver-debian .
\end{code}

\begin{code}{Extending Node 19 Buster (Debian)}{[without Entrypoint]}{}
docker build -t <your tag> --build-arg BASE=node:19-buster \
    --target=resolver-debian-bare .
\end{code}

\subsection{Extended Image Build Customizations}

There are sub-directories in the build-environment toolkit that are used as part of Building
the extended image.  Items can be add or modified in these directories as appropriate.

\subsubsection{CA Certificates}

The \texttt{cacerts} directory contains Amazon AWS Root CA certs that are used as the CA 
certificate for Checkmarx services.  You can add your own PEM encoded CA certificates in this
directory and it will be included as a trusted CA by the image.

If desired, you can remove the AWS CA files as long as there is at least one PEM
encoded certificate left in the directory during image build.

\subsubsection{\scaresolver Configuration YAML}

The \texttt{default-config} folder contains the \texttt{Configuration.yml} file with a default
configuration for \scaresolver.  It is possible to modify the default configuration so that
common parameter values are not needed to be provided for every invocation of \scaresolver.

\subsection{Dockerfile Targets with Entrypoints}\label{ssec:entrypoint_targets}

If you want to invoke \scaresolver in the extended image in the same way it is invoked from the command
line if it were locally installed, use the entrypoint targets.  If you intend to use the
extended images with the \cxtoolkit 
tools for \hyperref[chap:build_env_affinity]{webhook} scan workflows, extended
images with entrypoints are required.  The current targets that build the extended
image with an entry point are:

\begin{itemize}
    \item \texttt{resolver-alpine}
    \item \texttt{resolver-debian}
    \item \texttt{resolver-redhat}
    \item \texttt{resolver-amazon}
\end{itemize}


\subsection{Bare Dockerfile Targets}\label{ssec:bare_targets}

Containers built with bare targets have no entrypoint and run as root.  
Some CI/CD pipelines will need the ability to execute environment
configuraton commands as root before the stage is executed.  Some CI/CD pipelines
will allow the entrypoint to be overridden and will successfully execute the stage
in the container image.  The Azure Devops pipeline, for example, is not compatible
with images that define an entrypoint.  If your CI/CD pipeline needs a container image without 
an entrypoint, these targets will
produce extended images without an entrypoint:

\begin{itemize}
    \item \texttt{resolver-alpine-bare}
    \item \texttt{resolver-debian-bare}
    \item \texttt{resolver-redhat-bare}
    \item \texttt{resolver-amazon-bare}
\end{itemize}


\section{Extended Containers and Execution Sandboxing}

Building extended images with entrypoint targets will "sandbox" \scaresolver execution 
to the extent possible. This is to allow \scaresolver to execute a dependency resolution
while minimizing the risk of detonating malware payloads found in untrusted build scripts or 
package installation scripts. This is not universally a problem with all dependencies, but 
there is always the possibility for malware delivery via open-source packages. 

\noindent\\In terms of sandboxing, the extended images with entrypoints perform the following sandboxing
activities:

\begin{itemize}
    \item The local user executing the scan has limited privileges.
    \item Code for scanning is provided in a read-only volume map.
    This blocks dependencies that execute code-modifying malware from mutating scanned code.
    \item Output volume maps are write-only, preventing the vulnerability reports and logs
    from being exfiltrated as part of exploitable vulnerability intelligence gathering activities.
\end{itemize}

\noindent\\Nothing is foolproof; don't expect that using this container alone hardens your build
environment.  A threat modeling exercise should be undertaken to understand
if there are other infrastructure changes needed to properly control what is executed in your
build environments.

\subsection{Invoking the Sandbox Container CLI Style}\label{ssec:invoking_cli}

The extended container can invoke \scaresolver or \cxonecli in the same way each CLI tool would be
invoked locally.  Since the container is not executing locally, the code artifacts under scan
need to be mapped to paths inside the container.  Table \ref{table:volume_maps} shows the
container paths where it is appropriate to map volumes during container execution.  

While it is possible
to define your own local container paths for mapping, the paths in Table \ref{table:volume_maps}
have been configured with the container's execution user's permissions set to limit the ability to
interact with the paths as appropriate for the path's purpose.  The \texttt{/sandbox/input} path,
for example, is read-only to prevent modification of the code under scan.

Extended containers with no entrypoint (e.g. the "bare" targets) have the same permissions set
as would be used by extended containers with an entrypoint.  The no-entrypoint targets, however, run
as \texttt{root} which makes the permissions irrelevant.  It is possible to execute the CLI tool as
the sandbox user if desired.  Refer to Appendix \ref{chap:build_args} to see how to detect the user id
of the sandbox user.



\begin{table}[h]
    \caption{Container Volume Mapping Paths}\label{table:volume_maps}      
    \begin{tabularx}{\textwidth}{lcl}
        \toprule
        \textbf{Container Directory} & \textbf{Required} & \textbf{Description}\\
        \midrule
        \texttt{/sandbox/scalogs} & N & \makecell[l]{Used to write \scaresolver logs.}\\
        \midrule
        \texttt{/sandbox/input} & Y & \makecell[l]{This is where the input should be mapped\\
        for \scaresolver inputs.}\\
        \midrule
        \texttt{/sandbox/output} & Y & \makecell[l]{This is the directory where \scaresolver\\
        results files will be written.}\\
        \midrule
        \texttt{/sandbox/report} & Y & \makecell[l]{This is the directory where \scaresolver\\
        report will be written.}\\
        \bottomrule
    \end{tabularx}
\end{table}



\subsubsection{Invoking SCA Resolver}

Extended images with entrypoints can be invoked to use \scaresolver the same way
it would be invoked from the command line with a local install.  Any of the 
\scaresolver
\href{https://checkmarx.com/resource/documents/en/34965-132888-checkmarx-sca-resolver-configuration-arguments.html}{command line arguments}
are passed to the container.  A typical execution is shown below:


\begin{code}{Extended Container Typical CLI Invocation}{[SCA Resolver]}{}
docker run --rm -it \
    -v /my-log-path:/sandbox/scalogs \
    -v .:/sandbox/input \
    -v ./sca-results:/sandbox/output \
    my-container-tag \
    --logs-path /sandbox/scalogs \
    --resolver-result-path /sandbox/output \
    --scan-path /sandbox/input \ 
    <other SCAResolver args...>
\end{code}

Note that options passed to \scaresolver that indicate where to place output should be provided 
with paths prefixed with \texttt{/sandbox} corresponding to the local container paths in 
Table \ref{table:volume_maps}.

\subsubsection{Invoking CxOne CLI with SCA Resolver}

Invoking \scaresolver as a CLI with the extended container is performed by default with the entrypoint
of the extended container.  Using \texttt{cxone} as the first parameter to the extended container will
execute the \cxonecli as if it were invoked from the command line with a local install. A typical
execution is shown below:\\


\begin{code}{Extended Container Typical CLI Invocation}{[CxOne CLI with SCA Resolver]}{}
    docker run --rm -it \
        -v /my-log-path:/sandbox/scalogs \
        -v .:/sandbox/input \
        -v ./sca-results:/sandbox/output \
        my-container-tag \
        cxone \
        scan create \
        --output-path /sandbox/output \
        --sca-resolver /sandbox/resolver/ScaResolver \
        -s /sandbox/input \
        <other CxOne CLI args...>
\end{code}

Note that the \texttt{--sca-resolver} parameter is the container local path where the \scaresolver
executable can be found.

\chapter{Executing Supply Chain Scans with a Web Hook}\label{chap:build_env_affinity}


\section{Supply Chain Vulnerability Scans with \cxone}

Webhook integrations with \cxone are provided via the feedback applications.
Integration of \scaresolver for local execution is not currently available in
\cxone using webhooks via the feedback applications.  


\section{Supply Chain Vulnerability Scans with CxFlow++}

\cxflow is a scan orchestration tool that is used for orchestrating scans for \cxsast and \cxsca.  One
of the typical methods of invoking scans is to received SCM webhook payloads to indicate when
a push or pull-request event occurs on a source code repository.  The required scan types are orchestrated,
and the results are posted to a pull request comment or issues tracker tickets are opened.

\cxflowplusplus is a re-packaged container image of \cxflow.  The configuration of \cxflow
does not change, but \cxflowplusplus adds some additional capabilities:

\begin{itemize}
    \item Advanced configuration options allow for pre-execution download of \cxflow configuration
    artifacts.
    \item Supply chain vulnerability scans via \scaresolver can be performed using
    \hyperref[sec:extending_environment]{build environment extension} container images with 
    affinity to the code targeted for the scan.
\end{itemize}

Figure \ref{fig:dispatcher_workflow} depicts the workflow of the dispatcher as it resolves
the \scaresolver execution environment that has affinity to a project with scans
orchestrated by \cxflowplusplus.  The scans are initiated by a web hook payload sent to the
\cxflowplusplus endpoint.  \cxsast scans are orchestrated in parallel with \cxsca scans by
\cxflowplusplus with the additional logic in the dispatcher.

Projects can be tagged to allow dispatcher to locate the correctly configured \scaresolver
container, as is demonstrated with Applications A and B in Figure \ref{fig:dispatcher_workflow}.
Application C and D may not be tagged for affinity with a specific container, so the
\scaresolver scan would be executed in a container configured with the \texttt{Default} tag.

\begin{figure}[h]
    \caption{\cxflowplusplus Dispatcher Workflow}
    \includegraphics[width=\textwidth]{graphics/dispatcher_workflow.png}
    \label{fig:dispatcher_workflow}
\end{figure}


Some operational aspects of \cxflow may change when using \cxflowplusplus.  The \cxflowplusplus 
build environment affinity feature
\footnote{If there is no need to use this features of the dispatcher, \cxflowplusplus may not offer any advantages over \cxflow.} 
utilizes
\href{https://www.docker.com/blog/docker-can-now-run-within-docker/}{docker-in-docker (DIND)}; this may
change the compatibility of the runtime environment for an existing \cxflow deployment.

\subsection{\cxflowplusplus Runtime Environment}

The minimum recommended environment for \cxflowplusplus is:

\begin{itemize}
    \item 4 Cores
    \item 32 GB RAM
    \item 200GB of disk space
    \item Linux OS with kernel >= 5.12 if using \sysbox
\end{itemize}

\cxflowplusplus will be loading and executing container images as part of supply chain vulnerability scan
orchestration.  This can be potentially CPU and memory intensive.  Vertical or horizontal scaling
of \cxflowplusplus may be required to be able to orchestrate the volume of scans requested.

\noindent\\The execution of \cxflowplusplus must have some additional parameters provided to the
\texttt{docker} command to be able to create and execute docker containers inside the \cxflowplusplus
container.  If these additional parameters are not provided, \cxflowplusplus will emit errors when
attempting to execute docker containers inside the running container.

\subsubsection{DIND - Privileged Execution}

One method of enabling docker-in-docker is to execute containers in privileged mode.
Running \cxflowplusplus in privileged mode can be done like so:


\noindent\\\texttt{docker run -privileged -d ----restart unless-stopped \textbackslash }
\noindent\\ \tabto{5mm} \texttt{\cxflowplusplustag}

\noindent\\\textbf{WARNING}
\noindent\\It is generally considered a bad security practice to run a docker container in privileged mode.
Given that some open source packages can execute malware payloads during dependency resolution, it is not
advised to run in privileged mode for production purposes.

\subsubsection{DIND - \sysbox Runtime}

An alternative to executing docker in privileged mode is to use \sysbox as a runtime.  \sysbox runs on Linux
and has a requirement of a kernel version >= 5.12.  Appendix \ref{chap:sysbox_install} has
a procedure that can be used to install \sysbox manually on a Linux OS.

\noindent\\\texttt{docker run ----runtime=sysbox-runc -d ----restart unless-stopped \textbackslash}
\noindent\\ \tabto{5mm} \texttt{\cxflowplusplustag}

\subsection{Logging}

\cxflow logs to the console to maintain running compatibility with the official \cxflow image.  
Any supply chain vulnerability scan environment affinity operations are logged in files found
on the container in the directory \texttt{/var/log/dispatcher}.  To maintain the logs across
restarts of the container, it is recommended to map a local volume to \texttt{/var/log/dispatcher}.
The use of logging files for the dispatcher is to avoid the console logs from being corrupted
by spontaneous log emissions from the dispatcher.


\subsection{\cxflow Image Compatibility}

All the existing Spring Boot facilities for configuring \cxflow via environment variables work 
with the \cxflowplusplus image.  Using the \cxflowplusplus image in place of the official
\cxflow image is compatible as long as there is no need to execute \scaresolver.

If using \scaresolver, more configuration needed to provide a method of container image selection
for use in supply chain vulnerability scans.  The container will actively prevent
configuration of the \cxflow \texttt{SCA\_PATH\_TO\_SCA\_RESOLVER} option.  

The \cxflowplusplus image does not contain the \scaresolver runtime that would normally be found
in the official \cxflow image.  The \cxflowplusplus image replaces \scaresolver with
a mediation script known as \textit{the dispatcher} that invokes \scaresolver in an appropriate
container image.

The \cxflowplusplus image supports environment variables that change how it operates when 
preparing to start the \cxflow services.  Appendix \ref{chap:image_opts} has a complete list
of \cxflowplusplus additional options.


\subsection{Config-as-Code Build Environment Affinity}
When an supply chain vulnerability scan is invoked, the \texttt{default} 
image tag\footnote{The image tag refers to the tag of specified in the dispatcher configuration.  This should not be confused with the container tag.}
is used unless a config-as-code file is provided that explicitly defines the
image tag to use for the execution environment.

A config-as-code file named \texttt{.cxsca} should be placed in the root of the repository.  
The content of the file is JSON with the following structure:
\\
\begin{code}{.cxsca}{Config-as-Code Structure}{}
{
    "version" : "1",
    "tag" : "<dispatcher image tag>"
}
\end{code}

\subsection{\cxflowplusplus Dispatcher Configuration}

The dispatcher can configuration can be represented by either or both a YAML definition file and 
Environment variables.  The first YAML file found in \texttt{/dispatcher/yaml} when 
the dispatcher starts is used as the configuration file.  The configuration is 
evaluated in the following order of precedence:

\begin{enumerate}
\item Environment variables
\item YAML configuration
\end{enumerate}

\noindent\\This allows for environment variables to override YAML configurations or to provide
a combination of YAML and environment variable configured options.  A YAML configuration
file can be provided at runtime by mapping a local volume containing the configuration YAML to 
\texttt{/dispatcher/yaml}.  A YAML configuration file can alternately be provided at runtime
using the \hyperref[sec:DISPATCHERYAMLURL]{\texttt{DISPATCHER\_YAML\_URL}} \cxflowplusplus startup
environment variable. 

\noindent\\The complete reference for configuring the dispatcher can be found in
Appendix \ref{chap:dispatcher_config}.





\part{Appendices}
\appendix
\chapter{Release Notes}

\let\sectionold\thesection

\renewcommand\thesection{v}


\section{1.3}

\begin{itemize}
    \item Added the autobuilder.sh script to the build-environment.
    \item Release name of the build-environment now more suitable for automated download of latest release.
\end{itemize}


\section{1.2}

\begin{itemize}
    \item Fix for issue \#39: \textit{Appendix B.4 should document GROUP\textunderscore ID}
    \item Feature \#38: \textit{Integrate CxOne CLI}
\end{itemize}

\section{1.1}

\begin{itemize}
    \item Documentation updates.
    \item Publish a tarball for the CxFlow++ docker image.
\end{itemize}

\section{1.0}

\subsection*{Initial Release}


\let\thesection\sectionold

\chapter{Build Environment Container Build Arguments}\label{chap:build_args}

\newcommand{\buildarg}[3]{
    \section{#3}\label{sec:#1}
    Usage: \texttt{----build-arg #3=#2}\\
    }

\buildarg{BASE}{<base container image:tag>}{BASE}

The build argument \texttt{BASE} is optional and will default to \texttt{alpine:latest} if not 
provided.  The \texttt{BASE} argument should be the tag of a container image that you have 
defined as appropriate for the \texttt{--target} build stage. Your base image should 
contain all required build tooling that would be used in resolving dependencies when run 
against a project that would normally build with that base image.

\buildarg{CONFIGDIR}{<directory relative to the dockerfile>}{CONFIG\textunderscore DIR}

The build argument \texttt{CONFIG\textunderscore DIR} is optional and will default to \texttt{default-config} 
if not provided.  You can create your custom \scaresolver configuration YAML by creating a 
new directory and copying the contents of \texttt{default-config} to initialize the configuration 
artifacts.  You can then modify the configuration artifacts to fit the configuration needed 
by the generated image.


\input{sections/appendix/build-environment-args/USER_ID.tex}




\buildarg{USERID}{<container host invoking user's user id>}{USER\textunderscore ID}

Defaults to "1000", which is generally the ID of the first non-system group created on a clean
Linux system.  Most CI/CD tools will invoke pipeline stages (and thus container images) from
a user account the defined by the pipeline infrastructure.  Since container host directories or files
mapped to a container volume maintain the user/group ownership ID and permission bits when mapped, 
the container's user/group IDs used to set ownership on the \texttt{/sandbox} directory contents
need to match the IDs in the container host.

At build time, a group with the specified ID will be created if it doesn't exist.  If your base 
container already has the group with the specified ID, the group will be used as the group that 
executes \scaresolver. To detect if you need to change this value, you can execute the following 
command in a stage in your pipeline to 
\href{https://www.cyberciti.biz/faq/understanding-etcpasswd-file-format/}{see the group ID}
of the user running the pipeline steps:


\noindent\\\texttt{getent group \$(groups)}






\end{document}