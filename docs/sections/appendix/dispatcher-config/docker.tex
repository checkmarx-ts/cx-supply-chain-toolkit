\section{\texttt{docker}}

This section is optional.  It is intended to provide defaults for Docker invocations.

\subsection{\texttt{docker.login}}

This section is a list of container repositories where a \texttt{docker login} command will be 
executed on \cxflowplusplus startup.  If the images tags configured in the \texttt{resolver}
section are not stored in the default Docker public repository, configure the private 
image repositories here.

\noindent\\Multiple image repositories can be defined using multiple entries under 
\texttt{docker.login}.  Each key name under \texttt{docker.login} is the name of the
repository host.  Each repository host key is expecting a \texttt{username} and
\texttt{password} key/value pair.

\noindent\\Example YAML:\\

\begin{code}{\texttt{docker.login}}{YAML Structure}{}
docker:
    login:
        registry-1.docker.io:
            username: XXXXXXXXX
            password: XXXXXXXXX
        ghcr.io:
            username: XXXXXXXXX
            password: XXXXXXXXX
        my-host.com:
            username: XXXXXXXXX
            password: XXXXXXXXX
\end{code}

\subsubsection{\texttt{docker.login} as Environment Variables}

Dots in hostnames are represented in variable names by a single underscore.  Dashes in hostnames can be represented 
in environment variable names with a double underscore.  
Underscores are not valid in DNS hostnames and are therefore not supported.

\noindent\\The above example of the \texttt{docker.login} YAML 
has the equivalent environment variables:\\

\begin{code}{\texttt{docker.login}}{Environment Variables}{}
DOCKER_LOGIN_REGISTRY__1_DOCKER_IO_USERNAME=XXXX
DOCKER_LOGIN_REGISTRY__1_DOCKER_IO_PASSWORD=XXXX
DOCKER_LOGIN_GHCR_IO_USERNAME=XXXX
DOCKER_LOGIN_GHCR_IO_PASSWORD=XXXX
DOCKER_LOGIN_MY__HOST_COM_USERNAME=XXXX
DOCKER_LOGIN_MY__HOST_COM_PASSWORD=XXXX
\end{code}
