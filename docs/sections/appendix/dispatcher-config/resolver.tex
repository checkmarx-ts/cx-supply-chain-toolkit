\section{\texttt{resolver}}

This section is used to define parameters for \scaresolver invocation.

\subsection{\texttt{resolver.twostage}}

This is optional; the default is \texttt{true}.

\noindent\\If \texttt{true}, an \texttt{online} \scaresolver scan will be split into two stages:\\

\begin{enumerate}
    \item The first stage is an \texttt{offline} scan.  When invoking the offline scan, 
    credentials for the SCA server are stripped from the command.  This executes the dependency
    resolution without leaving credentials exposed to any scripts that execute as part of the 
    dependency resolution.
    \item The second stage is an \texttt{upload} where the dependency resolution results are
    uploaded to the SCA server.
\end{enumerate}

\noindent\\If set to \texttt{false}, an \texttt{online} scan will be invoked in a single step.

\noindent\\The environment variable \texttt{RESOLVER\_TWOSTAGE} is equivalent to the YAML configuration.


\subsection{\texttt{resolver.defaults}}

This section is optional; it allows for overriding the hard-coded default values if desired.

\noindent\\The timespan values here can be represented by simple value strings with numeric 
indicators indicating time units such as \texttt{h} for hours, \texttt{m} for 
minutes, and \texttt{s} for seconds.  Example:

\noindent\\\texttt{2h15m30s} means 2 hours, 15 minutes, and 30 seconds

\noindent\\\texttt{15m} means 0 hours, 15 minutes, and 0 seconds

\noindent\\An example of the \texttt{resolver.defaults} section can be observed below:\\

\begin{code}{\texttt{resolver.defaults}}{YAML Structure}{}
resolver:
    defaults:
        containerttl: 15m
        exectimeout: 10m
        defaulttag: default
        delete: False
\end{code}


\noindent\\\texttt{resolver.defaults.containerttl} - If not provided, this defaults to 1 hour.  
This is the length
of time a container image will be used after the initial 
\texttt{docker pull} command is executed to download the image.  This allows updated images to be 
retrieved without the need to restart \cxflowplusplus.

\noindent\\\texttt{resolver.defaults.exectimeout} - If not provided, this defaults to 30 minutes.  
This is the amount of time \cxflowplusplus will allow the \scaresolver image to execute before 
killing it and assuming failure.

\noindent\\\texttt{resolver.defaults.defaulttag} - If not provided, this defaults to the value 
of \texttt{default}.  This is the tag of the image configured in \texttt{resolver.images} used if 
an unknown tag is requested.

\noindent\\\texttt{resolver.defaults.delete} - If not provided, this defaults to the value 
of \texttt{true}.  Setting this to \texttt{false} will prevent the container image created
for the scan from being deleted.  The \texttt{false} setting is only recommended for
troubleshooting purposes;
having the \scaresolver containers remain on the \cxflowplusplus runtime container will consume a
significant amount of disk space over time.

\subsubsection{\texttt{resolver.defaults} as Environment Variables}

The above example of the \texttt{resolver.defaults} YAML has the equivalent environment variables:

\begin{code}{\texttt{resolver.defaults}}{Environment Variables}{}
RESOLVER_DEFAULTS_CONTAINTERTTL=15m
RESOLVER_DEFAULTS_EXECTIMEOUT=10m
RESOLVER_DEFAULTS_DEFAULTTAG=default
RESOLVER_DEFAULTS_DELETE=False
\end{code}

\subsection{\texttt{resolver.images}}

This section is where the image parameters are set for a container matching a tag that indicates 
which container image to use when invoking \scaresolver.

Each dictionary of key/value pairs configured under the \texttt{resolver.images} uses the key value 
as the image tag.  The listing below shows an example of a configuration for images 
tagged \texttt{default} and \texttt{node}.  The containers defined would be instances of 
an \scaresolver image derived from an appropriate build environment base image.

In each of the sections, the \texttt{containerttl} and \texttt{exectimeout} values are optional.  
If not provided, the corresponding values from the \texttt{resolver.defaults} section will be used.\\


\begin{code}{\texttt{resolver.images}}{YAML Structure}{}
resolver:
    images:
        default:
            container: my-default-container:latest
            containerttl: 5m
            exectimeout: 10m
            execenv:
                FOO: BAZ
            execparams:
                - --log-level=Verbose
        node: 
            container: my-node-container:latest
            execenv:
                FOO: BAR
            envpropagate:
                - MY_ENVIRONMENT_VARIABLE1
                - MY_ENVIRONMENT_VARIABLE2
\end{code}



\noindent\\\texttt{resolver.images.<tag>.container} - (\textbf{Required}) The container image tag 
name associated with the configuration tag that will be resolved by the dispatcher
during the \cxflowplusplus scan execution.  

\noindent\\\texttt{resolver.images.<tag>.containerttl} - (\textbf{Optional})  Uses the 
corresponding value from \texttt{resolver.defaults} if not provided.

\noindent\\\texttt{resolver.images.<tag>.exectimeout} - (\textbf{Optional})  Uses the corresponding 
value from \texttt{resolver.defaults} if not provided.

\noindent\\\texttt{resolver.images.<tag>.execenv} - (\textbf{Optional}) A dictionary of key/value 
pairs that are emitted in the container's environment when the container image is executed.  
Entries with key values that match keys defined in \texttt{resolver.images.<tag>.envpropagate} 
will have the value overwritten by the value found in the propagated environment value.

\noindent\\\texttt{resolver.images.<tag>.envpropagate} - (\textbf{Optional}) An array of names of 
environment variables in the \cxflowplusplus environment that will be propagated as-is to the 
executing image environment.

\noindent\\\texttt{resolver.images.<tag>.execparams} - (\textbf{Optional}) An array of values passed
to \scaresolver at the end of all other parameters needed to control the
execution of \scaresolver.  The values here are mainly intended to pass configuration
values to dependency resolution tools invoked by \scaresolver.  Passing other values
to \scaresolver may cause operational conflicts.

\noindent\\\texttt{resolver.images.<tag>.dockerparams} - (\textbf{Optional}) A key/value dictionary 
that follows the \texttt{**kwargs} key/value pairs supported in the 
\href{https://docker-py.readthedocs.io/en/stable/containers.html#docker.models.containers.ContainerCollection.run}{Docker Python API} 
\texttt{run} method.

\subsubsection{\texttt{resolver.images} as Environment Variables}

\noindent\\The above example of the `resolver.images` YAML has the equivalent environment variables:\\

\begin{code}{\texttt{resolver.images}}{Environment Variables}{}
RESOLVER_IMAGES_DEFAULT_CONTAINER=my-default-container:latest
RESOLVER_IMAGES_DEFAULT_CONTAINERTTL=5m
RESOLVER_IMAGES_DEFAULT_EXECTIMEOUT=10m
RESOLVER_IMAGES_DEFAULT_EXECENV_FOO=BAZ
RESOLVER_IMAGES_DEFAULT_EXECPARAMS_1=--log-level=Verbose
RESOLVER_IMAGES_NODE_CONTAINER=my-node-container:latest
RESOLVER_IMAGES_NODE_EXECENV_FOO=BAR
RESOLVER_IMAGES_NODE_ENVPROPAGATE_1=MY_ENVIRONMENT_VARIABLE1
RESOLVER_IMAGES_NODE_ENVPROPAGATE_2=MY_ENVIRONMENT_VARIABLE2
\end{code}