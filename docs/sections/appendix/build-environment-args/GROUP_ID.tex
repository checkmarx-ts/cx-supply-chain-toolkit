



\buildarg{GROUPID}{<container host invoking user's group id>}{GROUP\textunderscore ID}

Defaults to "1000", which is generally the ID of the first non-system group created on a clean
Linux system.  Most CI/CD tools will invoke pipeline stages (and thus container images) from
a user account the defined by the pipeline infrastructure.  Since container host directories or files
mapped to a container volume maintain the user/group ownership ID and permission bits when mapped, 
the container's user/group IDs used to set ownership on the \texttt{/sandbox} directory contents
need to match the IDs in the container host.

At build time, a group with the specified ID will be created if it doesn't exist.  If your base 
container already has the group with the specified ID, the group will be used as the group that 
executes \scaresolver. To detect if you need to change this value, you can execute the following 
command in a stage in your pipeline to 
\href{https://www.cyberciti.biz/faq/understanding-etcpasswd-file-format/}{see the group ID}
of the user running the pipeline steps:


\noindent\\\texttt{getent group \$(groups)}


