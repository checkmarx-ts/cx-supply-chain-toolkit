\chapter{\cxtoolkit Maintainer's Guide}\label{chap:maintainers_guide}


This section is primarily for the maintainers of the \cxtoolkit.  It is intended
to document any aspects needed to publish a successful release build.


\section{GitHub Action Configuration Pre-Requisites}

Some GitHub action variables and secrets are required to be defined in the repository
for the workflows to work.  Anyone that forks the repository and wants to execute
the workflows will need to define these elements if they wish to test the workflows.

\subsection{Action Variables}

Action variables do not store sensitive information.  Please ensure any additional
variables added never need to store sensitive information.

\subsubsection{Variable: WORKFLOW\_BUILD\_COMPAT}

This is primarily for executing container build unit tests.  These tests will typically
run on a developers desktop without the need for this variable.  The variable adds options
to the \texttt{docker build} command so that it is compatible with the GitHub workflow
execution environment.  It is currently defined as:

\noindent\\\texttt{----load ----build-arg GROUP\_ID=127 ----build-arg USER\_ID=1001}

\noindent\\The UID/GID used by GitHub-hosted runners may periodically change. If 
there are build failures, check that the runner's user has the correct UID/GID.

\subsection{Action Secrets}

The secrets are used in the GitHub action workflows and are used for various functions
that execute inside the workflow.  Some of these values may be required to run some
of the container build unit tests.

\subsubsection{Secret: SCA\_TENANT}
This is a \cxsca tenant name used to perform unit tests.  If this is not provided,
the unit tests that perform scans will be skipped.

\subsubsection{Secret: SCA\_USER}
This is a user in the \cxsca tenant used to perform unit tests.  If this is not provided,
the unit tests that perform scans will be skipped.

\subsubsection{Secret: SCA\_PASSWORD}
This is the password for a user in the \cxsca tenant used to perform unit tests.  
If this is not provided,
the unit tests that perform scans will be skipped.

\subsubsection{Secret: PACKAGE\_USER}
This is the GitHub user that is used for performing workflows that
publish releases and packages.  If this is not provided, the workflow actions
will fail.

\subsubsection{Secret: PACKAGE\_PAT}

This is a PAT associated with the GitHub user defined as the secret
\texttt{PACKAGE\_USER}. If this is not provided, the workflow actions
will fail.

\section{Publishing Releases}

Releases and Pre-Releases are published by workflows found in the GitHub Action
tab on the repository.  The Action tab requires the appropriate permissions
to view; if you're not a member of the maintainer team for \cxtoolkit, you will
not see the action tab.

\subsection{Publish a Release or a Pre-Release}

Publishing a release is as simple as selecting the \texttt{Publish a Release}
workflow in the GitHub Actions tab.  To the right of the screen you will see
a "Run Workflow" button.  When clicked, you will have some options to configure:

\begin{enumerate}
    \item \textbf{Branch} - Select a branch as the source of the release generation.  For
    releases, this should be \texttt{master}.  Pre-releases can be generated from
    any branch as needed.
    \item \textbf{Version tag} - Provide a release version in the form of "x.x".  This value
    will be used to tag the repository and provide version identification information
    for release artifacts.
    \item \textbf{Prelease} - A checkbox that, if checked, will generate a pre-release.
\end{enumerate}

There can be many pre-releases published for a version but only one release of a version.
The workflows will perform code tagging as appropriate for the type of release.  The
workflows will also attempt to validate that the source code is tagged properly before
executing and that tags are removed in the event of a workflow failure.

\subsubsection{Release Publishing Workflows}

A release workflow will fail if a release of the same version has already been
set as a tag in the repo.  While it is possible to work around this by manually
removing repo tags and releases, it is not advised to do so to re-publish a release.
If re-publication of a release is needed, please see Section \ref{sec:republish}


\subsubsection{Pre-Release Publishing Workflows}

Each pre-release will build a tag from the version for which it is a pre-release
and include the build workflow iteration number as part of the tag.  This will allow
for multiple pre-releases to be published for any given release version.

\subsection{Republishing a Release}\label{sec:republish}

Republication of a release is is similar to publishing a release;
all that is required is selecting the \texttt{Republish a Release}
workflow in the GitHub Actions tab.  To the right of the screen you will see
a "Run Workflow" button.  When clicked, you will have some options to configure:

\begin{enumerate}
    \item \textbf{Branch} - Select a branch as the source of the release generation.
    If the re-published release branch and the published release branch differ, things
    may not work so well.
    \item \textbf{Version tag} - Provide a release version in the form of "x.x".  
    This value will be used to fetch the code at the proper tag in the release
    repository.
    \item \textbf{Tag packages as "latest"} - This will set the package tag
    for any published images as \texttt{latest}. 
\end{enumerate}

Republication of a release is limited to full releases only.  The concept of
republishing a release is typically so that newer versions of \cxflow can be
re-packaged into \cxflowplusplus published container images.  Future packaged
components that work similarly will follow this same workflow re-publication
logic.  There are  some things to keep in mind when re-publishing a release:

\begin{itemize}
    \item A release tag must exist for the version being re-published.
    \item The code from the \cxtoolkit is not changed in the release since
    the packaging is performed using code at the release tag.
    \item It will be a good idea to only set the package tag of the latest
    release to \texttt{latest}.  This allows older versions of \cxtoolkit
    to be republished with newer versions of external components incorporated
    in published container images.
    \item While it is possible to set the \texttt{latest} tage on a republished
    image for an older version of the \cxtoolkit, users may have configurations
    that are not backwards compatible.  It is a good idea to avoid re-tagging
    older container images as \texttt{latest}.
\end{itemize}

\section{\cxflowplusplus Container Image Tagging}

The \cxflowplusplus container image is tagged with the \cxtoolkit release or pre-release
tag at the time of the publication.  An additional tag with the \cxtoolkit version
and the incorporated \cxflow version is also provided.  Users of version 1.1,
for example, need only reference the tag \texttt{v1.1} to get the latest version
of \cxflowplusplus.  If \cxflow has a new version released, a republished version 1.1
will retrieve the container image with the latest \cxflow.

It will generally be the case that testing will be performed with \cxflow
versions that are published as future releases.  It is not the intention of the
\cxtoolkit to publish a container image for each past historical version of \cxflow.
As new versions of the \cxtoolkit are released, they will incorporate the latest
version of \cxflow.

If a container image tag for a specific version of \cxflow incorporated into the 
\cxflowplusplus image is not available, please refer to Section \ref{sec:CXFLOWALTJARURL}
for options to retrieve and run a specific \cxflow release at startup.