\chapter{Executing Supply Chain Scans in a Pipeline}


\section{On-Demand-Install Execution Method}

Using this method, all required CLI components are downloaded and executed at the time
the pipeline is executed.  Some advantages of this method:


\begin{itemize}
    \item It always gets the latest version of CLI tools.
    \item There is no additional configuration artifacts needed to be able to
    perform a supply chain vulnerability scans.
\end{itemize}

\noindent\\Some disadvantages of this method:
\begin{itemize}
    \item The endpoint for the download may be unreachable, thus causing the build to fail.
    \item The steps to perform the download and install may need to be repeated
    in many different pipeline definitions.
\end{itemize}


\subsection{SCA Standalone}

This method invokes the \scaresolver in the pipeline script by following this
basic pattern:

\begin{enumerate}
    \item Download the \scaresolver from the Checkmarx website or
    an internal storage endpoint.
    \item Extract \scaresolver from the distribution tarball.
    \item Execute \scaresolver with command line parameters relevant to
    to the code that is the target of the scan.
\end{enumerate}


\subsection{Checkmarx One}

This method invokes the \scaresolver via invocation of the \cxonecli in the pipeline script by 
following this basic pattern:

\begin{enumerate}
    \item Download the \cxonecli from the Checkmarx website or
    an internal storage endpoint.
    \item Download the \scaresolver from the Checkmarx website or
    an internal storage endpoint.
    \item Extract \cxonecli from the distribution tarball.
    \item Extract \scaresolver from the distribution tarball.
    \item Execute \cxonecli with command line parameters relevant to
    to the code that is the target of the scan.  One parameter will include
    the path to \scaresolver.
\end{enumerate}



\section{Containerized Build Environment Execution Method}

Many CI/CD pipeline tools allow stages to define container images used for
executing a pipeline stage.  Using techniques discussed in
Section \ref{sec:extending_environment}, a container with \cxonecli and \scaresolver
can be executed in a pipeline stage. The CI/CD pipeline tool may use a container with an 
\hyperref[ssec:entrypoint_targets]{Entrypoint} or one that
is used by the \hyperref[ssec:bare_targets]{Bare} container build targets.

\subsection{Azure DevOps Examples}

The \hyperref[listing:ado_pipeline2]{Azure Devops Pipeline Example} below 
shows a pipeline stage executing \scaresolver in a containerized environment.
In the case of Azure DevOps, containers
\href{https://learn.microsoft.com/en-us/azure/devops/pipelines/process/container-phases?view=azure-devops#requirements}{are required to not define an entrypoint.}
By building a container extending an existing containerized build
environment with a \hyperref[ssec:bare_targets]{Bare} build target, Azure DevOps
pipeline scripts execute inside the container as demonstrated in the listing below:\\

\label{listing:ado_pipeline2}
\begin{code}{Azure Devops Pipeline Example}{[SCA Resolver]}{}
trigger:
    - master

pool:
    vmImage: ubuntu-latest

jobs:
    - job: SCAResolver
        pool:
            vmImage: ubuntu-latest

        container: 
            image: cxnleach/scaresolver-general-build-ado:latest

        steps:
            - script: /sandbox/resolver/ScaResolver -h
\end{code}

Note that since the execution is occurring inside the
\texttt{cxnleach/scaresolver-general-build-ado:latest} container, the path to \scaresolver
is local to the container image.  The listing below shows a scan executing with the \cxonecli
also in Azure DevOps:\\


\label{listing:ado_pipeline_cxone}
\begin{code}{Azure Devops Pipeline Example}{[CxOne CLI]}{}
trigger:
    - master

pool:
    vmImage: ubuntu-latest

jobs:
    - job: SCAResolver
        pool:
            vmImage: ubuntu-latest

        container: 
            image: cxnleach/scaresolver-general-build-ado:latest

        steps:
            - script: /sandbox/cxonecli/cx -h
\end{code}


\subsection{GitHub Actions Samples}

GitHub actions can use extended containers with \hyperref[ssec:entrypoint_targets]{entrypoints} but
will require the user to have a UID of 1001 and a group id of 127.  The listing below shows 
an example of how to build an extended container while setting the UID/GID as described
in Appendix \ref{chap:build_args}:\\

\begin{code}{GitHub Extended Container Build Example}{}{}
docker build -t your_tag_here \
    --build-arg BASE=gradle:8-jdk11-alpine \
    --target=resolver-alpine \
    --build-arg USER_ID=1001 \
    --build-arg GROUP_ID=127 \
    -f Dockerfile .    
\end{code}


Execution as a step in the GitHub action yaml would be invoked as described in 
Section \ref{ssec:invoking_cli}, such as is demonstrated by the basic shell invocation pipeline
example shown below:\\

\begin{code}{Invoking the Container in a GitHub Action}{}{}
name: Checkmarx Scan

on:
    push:
        branches: [ $default-branch ]
    pull_request:
        branches: [ $default-branch ]

jobs:
    build:
        runs-on: ubuntu-latest

    steps:
        - uses: actions/checkout@v3
        - name: Execute a Checkmarx One Scan
            run: |
                docker run --rm -it \
                -v /my-log-path:/sandbox/scalogs \
                -v .:/sandbox/input \
                -v ./sca-results:/sandbox/output \
                my-container-tag \
                cxone \
                scan create \
                --output-path /sandbox/output \
                --sca-resolver /sandbox/resolver/ScaResolver \
                -s /sandbox/input \
                <other CxOne CLI args...>            
    
\end{code}
