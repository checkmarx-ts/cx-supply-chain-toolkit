\chapter{Executing Supply Chain Scans in a Pipeline}


\section{Supply Chain Scans with \cxone}

This document does not currently cover integration topics for integrating
\scaresolver invoked in a pipeline with \cxone.  This will be covered in 
future releases.


\section{\scaresolver On-Demand-Install Execution Method}

This method invokes the \scaresolver in the pipeline script by following this
basic pattern:

\begin{enumerate}
    \item Download the \scaresolver from the Checkmarx website or
    an internal storage endpoint.
    \item Extract \scaresolver from the distribution tarball.
    \item Execute \scaresolver with command line parameters relevant to
    to the code that is the target of the scan.
\end{enumerate}

Some advantages of this method:

\begin{itemize}
    \item It always gets the latest version of \scaresolver.
    \item There is no additional configuration artifacts needed to be able to
    perform a supply chain vulnerability scan.
\end{itemize}



Some disadvantages of this method:
\begin{itemize}
    \item The endpoint for the download may be unreachable, thus causing the build to fail.
    \item The steps to perform the download and install may need to be repeated
    in many different pipeline definitions.
\end{itemize}



\section{Containerized Build Environment Execution Method}

Many CI/CD pipeline tools allow stages to define container images used for
executing a pipeline stage.  Using techniques discussed in
Section \ref{sec:extending_environment}, a container with \scaresolver
configured for execution can be used to execute a pipeline stage.  The
CI/CD pipeline tool may use a container with an 
\hyperref[ssec:entrypoint_targets]{Entrypoint} or one that
is used by the \hyperref[ssec:bare_targets]{Bare} container build targets.

The \hyperref[listing:ado_pipeline2]{Azure Devops Pipeline Example} below 
shows a pipeline stage executing \scaresolver in a containerized environment.
In the case of Azure DevOps, containers
\href{https://learn.microsoft.com/en-us/azure/devops/pipelines/process/container-phases?view=azure-devops#requirements}{are required to not define an entrypoint.}
By building a container extending an existing containerized build
environment with a \hyperref[ssec:bare_targets]{Bare} build target, Azure DevOps
pipeline scripts execute inside the container as demonstrated in the listing.

\label{listing:ado_pipeline2}
\begin{code}{Azure Devops Pipeline Example}{}{}
trigger:
    - master

pool:
    vmImage: ubuntu-latest

jobs:
    - job: SCAResolver
        pool:
            vmImage: ubuntu-latest

        container: 
            image: cxnleach/scaresolver-general-build-ado:latest

        steps:
            - script: /sandbox/resolver/ScaResolver -h
\end{code}
